% a0poster Portrait Poster
% LaTeX Template
% Version 1.0 (22/06/13)
%
% The a0poster class was created by:
% Gerlinde Kettl and Matthias Weiser (tex@kettl.de)
%
% This template has been downloaded from:
% http://www.LaTeXTemplates.com
%
% License:
% CC BY-NC-SA 3.0 (http://creativecommons.org/licenses/by-nc-sa/3.0/)
%

%----------------------------------------------------------------------------------------
%	PACKAGES AND OTHER DOCUMENT CONFIGURATIONS
%----------------------------------------------------------------------------------------

% \directlua{pdf.setminorversion(7)}
\documentclass[a0,portrait]{a0poster}

\usepackage{multicol} % This is so we can have multiple columns of text side-by-side
\columnsep=100pt % This is the amount of white space between the columns in the poster
% \columnseprule=3pt % This is the thickness of the black line between the columns in the poster

\usepackage[svgnames]{xcolor} % Specify colors by their 'svgnames', for a full list of all colors available see here: http://www.latextemplates.com/svgnames-colors

\usepackage{times} % Use the times font
%\usepackage{palatino} % Uncomment to use the Palatino font

\usepackage{graphicx} % Required for including images
\graphicspath{{figures/}} % Location of the graphics files
\usepackage{booktabs} % Top and bottom rules for table
\usepackage[font=small,labelfont=bf]{caption} % Required for specifying captions to tables and figures
\usepackage{amsfonts, amsmath, amsthm, amssymb} % For math fonts, symbols and environments
\usepackage{wrapfig} % Allows wrapping text around tables and figures
\usepackage{bm}
\usepackage[hidelinks]{hyperref}

\begin{document}

%----------------------------------------------------------------------------------------
%	POSTER HEADER
%----------------------------------------------------------------------------------------

% The header is divided into two boxes:
% The first is 75% wide and houses the title, subtitle, names, university/organization and contact information
% The second is 25% wide and houses a logo for your university/organization or a photo of you
% The widths of these boxes can be easily edited to accommodate your content as you see fit

\begin{minipage}[b]{0.75\linewidth}
 \VERYHuge \color{Black} \textbf{pyhf} \color{Black}\\[0.5cm] % Title
 \Huge\textbf{pure Python implementation of HistFactory}\\[1cm] % Subtitle
 % \author{{\href{https://www.matthewfeickert.com/}{\underline{Matthew Feickert}$^{1}$}}, \href{http://www.lukasheinrich.com/}{Lukas Heinrich$^{2}$}, \href{https://giordonstark.com/}{Giordon Stark$^{3}$}, \href{http://theoryandpractice.org/}{Kyle Cranmer$^{4}$}}
 % \institution{1 Southern Methodist University,~~~2 CERN,~~~3 University of California Santa Cruz,~~~4 New York University}
 \color{DarkSlateGray} % DarkSlateGray color for the rest of the content
 \large \textbf{{\href{https://www.matthewfeickert.com/}{\underline{Matthew Feickert}$^{1}$}}, \href{http://www.lukasheinrich.com/}{Lukas Heinrich$^{2}$}, \href{https://giordonstark.com/}{Giordon Stark$^{3}$}, \href{http://theoryandpractice.org/}{Kyle Cranmer$^{4}$}}\\[0.5cm] % Author(s)
 \normalsize {1 University of Illinois at Urbana-Champaign,~2 CERN,~3 University of California Santa Cruz,~4 New York University}\\ % University/organization
 % \Large \texttt{john@LaTeXTemplates.com} --- 1 (000) 111 1111\\
 \color{Black}
\end{minipage}
%
\begin{minipage}[b]{0.25\linewidth}
 \includegraphics[width=20cm]{pyhf-logo.png}\\
\end{minipage}

\vspace{1cm} % A bit of extra whitespace between the header and poster content

%----------------------------------------------------------------------------------------

\begin{multicols}{2} % This is how many columns your poster will be broken into, a portrait poster is generally split into 2 columns

 %----------------------------------------------------------------------------------------
 %	ABSTRACT
 %----------------------------------------------------------------------------------------

 % \color{Navy} % Navy color for the abstract

 %----------------------------------------------------------------------------------------
 %	INTRODUCTION
 %----------------------------------------------------------------------------------------

 % \color{SaddleBrown} % SaddleBrown color for the introduction

 \section*{HistFactory}
 One of the most widely used statistical models in \textbf{high energy physics} for binned measurements and searches

 \begin{center}\vspace{1cm}
  \includegraphics[width=\linewidth]{HistFactory_result_examples.png}
 \end{center}
 %
 \begin{minipage}{0.33\linewidth}
  \begin{center}
   \large\textbf{Standard Model}
  \end{center}
 \end{minipage}%
 \quad
 \begin{minipage}{0.33\linewidth}
  \begin{flushleft}
   \large\qquad\textbf{Supersymmetry}
  \end{flushleft}
 \end{minipage}%
 \quad
 \begin{minipage}{0.33\linewidth}
  \begin{flushleft}
   \large\quad\qquad\textbf{Exotics}
  \end{flushleft}
 \end{minipage}%

 \section*{Declarative binned likelihoods}

 \[
  f(\bm{n}, \bm{a} \,|\,\bm{\phi},\bm{\chi}) = \underbrace{\color{blue}{\prod_{c\in\mathrm{\,channels}} \prod_{b \in \mathrm{\,bins}_c}\textrm{Pois}\left(n_{cb} \,\middle|\, \nu_{cb}\left(\bm{\eta},\bm{\chi}\right)\right)}}_{\substack{\text{Simultaneous measurement}\\%
    \text{of multiple channels}}} \underbrace{\color{red}{\prod_{\chi \in \bm{\chi}} c_{\chi}(a_{\chi} |\, \chi)}}_{\substack{\text{constraint terms}\\%
    \text{for }\text{``auxiliary measurements''}}}
 \]

 \noindent\textcolor{blue}{Primary Measurement}:
 \begin{itemize}
  \item Multiple disjoint ``channels'' (e.g. event observables) each with multiple bins of data
  \item Example parameter of interest: strength of physics signal, $\mu$
 \end{itemize}
 \textcolor{red}{Auxiliary Measurements}:
 \begin{itemize}
  \item Nuisance parameters (e.g. in-situ measurements of background samples)
  \item Systematic uncertainties (e.g. normalization, shape, luminosity)
 \end{itemize}

 \section*{Performance}
 Efficient use of tensor computation makes pyhf fast
 \begin{center}
  \includegraphics[width=\linewidth]{performance_only.pdf}
 \end{center}
 Competitive with traditional \texttt{C++} implementation --- often faster

 \section*{Hardware Acceleration}
 For ML-library tensor backends the computational graph can be transparently placed on hardware accelerators: \textbf{GPUs} and \textbf{TPUs} for order of magnitude speed-up in computation
 \begin{center}
  \includegraphics[width=\linewidth]{scaling_hardware.pdf}
 \end{center}

 \section*{Implementation}
 \begin{center}
  \includegraphics[width=0.9\linewidth]{computational_graph3.pdf}
 \end{center}
 The computational graph of multidimensional array operations for likelihood function of a physics analysis defined through HistFactory

 %
 \vspace{0.5em}
 %
 \begin{minipage}{0.33\linewidth}
  \begin{center}
   \includegraphics[width=\linewidth]{NumPy_logo.pdf}
  \end{center}
 \end{minipage}%
 \quad
 \begin{minipage}{0.33\linewidth}
  \begin{center}
   \includegraphics[width=\linewidth]{TensorFlow_logo.pdf}
  \end{center}
 \end{minipage}%
 \quad
 \begin{minipage}{0.33\linewidth}
  \begin{center}
   \includegraphics[width=0.85\linewidth]{Pytorch_logo.pdf}
  \end{center}
 \end{minipage}%
 \vspace{0.25cm}

 Use of $n$-dimensional array (``tensor'') operations through a common API layer around high performance tensor libraries

 \section*{JSON Specification}
 The full likelihood can be expressed as a \textbf{single JSON document}\\
 Archive friendly for analysis presentation
 \vspace{0.5em}
 \begin{center}
  \includegraphics[width=0.9\linewidth]{carbon_JSON_spec.png}
 \end{center}
 \vspace{-1em}
 \begin{center}
  {\textbf{Example:} 2 binned single channel with 2 samples with 1 parameter of interest and 1 nuisance parameter}
 \end{center}
 \begin{center}
  \includegraphics[width=0.9\linewidth]{carbon_pyhf_CLs.png}
 \end{center}

 \begin{minipage}{0.58\linewidth}
  \begin{center}
   \href{http://iris-hep.org/}{\includegraphics[width=\linewidth]{IRIS-HEP_logo.eps}}
   % \href{http://iris-hep.org/}{\includegraphics[width=\linewidth]{IRIS-HEP_logo-eps-converted-to.pdf}}
  \end{center}
 \end{minipage}%
 \quad
 \begin{minipage}{0.38\linewidth}
  \begin{center}
   \href{https://pypi.org/project/pyhf/}{\includegraphics[width=\linewidth]{pyhf_PyPI.pdf}}
  \end{center}
  \vspace{1em}
  \begin{center}
   \href{https://doi.org/10.5281/zenodo.1169739}{\includegraphics[width=\linewidth]{zenodo_doi.pdf}}
  \end{center}
  \vspace{3em}
 \end{minipage}%

 % \color{DarkSlateGray} % DarkSlateGray color for the rest of the content

 % \begin{center}\vspace{1cm}
 %  \includegraphics[width=0.8\linewidth]{placeholder}
 %  \captionof{figure}{\color{Green} Figure caption}
 % \end{center}\vspace{1cm}
 %
 % \nocite{*} % Print all references regardless of whether they were cited in the poster or not
 % \bibliographystyle{plain} % Plain referencing style
 % \bibliography{sample} % Use the example bibliography file sample.bib
 %
 %
 % \section*{Acknowledgements}
 %
 % Etiam fermentum, arcu ut gravida fringilla, dolor arcu laoreet justo, ut imperdiet urna arcu a arcu. Donec nec ante a dui tempus consectetur. Cras nisi turpis, dapibus sit amet mattis sed, laoreet.

\end{multicols}
\end{document}
